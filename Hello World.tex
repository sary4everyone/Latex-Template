\documentclass[12pt]{article}  %other than aarticle we can use letter/book/report

\usepackage{amsmath}
\usepackage{amssymb}
\usepackage{amsfonts}
\usepackage{graphicx}
\usepackage{color}
%can use white, black, red, green, blue, cyan, magenta, yellow

%or we can use for more colours
%\usepackage[usenames,dvipsnames,svgnames,table]{xcolor}

\usepackage{fullpage}
%we can also use \usepackage[top=1in,bottom=1in,right=1in,left=1in]{geometry} 
%or use\usepackage[margin=1in,paperheight=11in,paperwidth=8.5in]{geometry} 

\def\EquationRepeat{y = mx + 3}   %using Macros

\begin{document}

\begin{titlepage}

\begin{huge}

\begin{center}

%\texttt{\underline{\textbf{\textit{\color{red}{The First Use of Latex While Learning It}}}}}

\end{center}



\end{huge}

\end{titlepage}

\renewcommand{\contentsname}{List of Contents}


\tableofcontents

\title{\color{green}{Hello World in \LaTeX \ Document}}
\author{\color{blue}{Sary}}
\date{\today}
\maketitle
%This line is must

\section{Text Part}
	\subsection{Font}
		\subsubsection{Font Size}
\begin{flushleft}
We experiment with style\
\end{flushleft}
Turns out it's fun actually.
\begin{center}
I have this in \textbf{boldface}\\
I have this in \textit{italic}\\
I have this \emph{emphasized}\\
\end{center}

\begin{flushright}
It's flushed right.
\end{flushright}

%This is used to \hspace{2in} give gap \\ %\vspace{} can also be used
 
\begin{center}
\begin{tiny}
Fontsize\\
\end{tiny}
\begin{small}
Fontsize\\
\end{small}
\begin{normalsize}
Fontsize\\
\end{normalsize}
\begin{Large}
Fontsize\\
\end{Large}
\begin{Huge}
Fontsize\\
\end{Huge}
\end{center}

This is the set of natural numbers $\mathbb{N}$

From preamble we have $\EquationRepeat$
\begin{center}
$(x+1)$ \\
$[x+1] $\\
$\{x+1\}$ \\
$\$50.50 $ was spent last day \\
$\left(\dfrac{x^2+x+3}{x+2}\right)$ \\
$\left|\dfrac{x^2+x+3}{x+2}\right|$\\
$
\left.\dfrac{dy}{dx}\right|_{x=0}
$
\end{center}


\begin{center}
$
\left.\dfrac{d^{2}y}{dx^{2}}\right|_{x=0}
$
\end{center}

I have

\begin{enumerate}
\item speed
\item stamina
\item talent
\end{enumerate}

I also have

\begin{itemize}
\item power
	\begin{itemize}
	\item high
	\item low
	\end{itemize}
\item energy
	\begin{itemize}
	\item[doggy] huge
	\item[meow]  tiny
\end{itemize}
\end{itemize}

\noindent Not indenting

Indenting


%\noindent Spacing\\


a b

a   b

a \, b

a \; b

a \quad b

a \vspace{2cm} b

a \vspace{3in} b



\begin{verbatim}
> x=rnorm(20, 1.1, 2.4)
> epsilon=rnorm(20, 0, 1.8)
> a=2.5
> b=3
> y=a+(b*x)+epsilon
> data<-cbind(x,y)
> data
               x          y
[1,]  3.3019975 10.5501417
[2,] -0.3212579  1.8841905
[3,]  2.8453046  9.8698746
[4,] -0.3486842 -0.8780308
[5,] -2.3812394 -1.7281941
\end{verbatim}


\begin{eqnarray*}
4x^2=& 12\\
x^2 =& 3\\
x   \approx& \pm 1.732
\end{eqnarray*}

\begin{large}
WOW
\end{large}

\begin{displaymath}
\psi(x) = \left\{
			\begin{array}{lr}
				
				\ \ x  : &  0 \leq x < \infty \\
			   -x : &  (-)\infty < x < 0   
				
			\end{array}		 
		 \right.			
\end{displaymath}


	\subsection{Maths}
Lets look at the following equation we solve in \LaTeX. 
We want to solve for real values of x in the following equation:
\[
2x^2+3(x-1)(x-2)=0
\]


Note that
\[
\dfrac{3}{4}-\dfrac{6}{8}=0
\]


\begin{align*}
2x^2 + 3(x-1)(x-2) &= 2x^2 + 3(x^2 - 3x + 2)\\
				   &= 2x^2 + 3x^2 - 9x + 6\\
				   &= 5x^2 - 9x + 6
\end{align*}
\[
5x^2 - 9x + 6 = 0
\]
Solving we have:

\begin{align*}
x &= \dfrac{9 ^{+}_{-} \sqrt{81 - 120}}{10}\\
  &= \dfrac{9 ^{+}_{-} \sqrt{39}i}{10}
\end{align*}
\[
x^3 + y^3 = 0
\]

Recall from previous day

\begin{equation}
x^2 + y^2 = 0 
\end{equation}
\begin{equation}
x^3 + y^3 = 0 
\end{equation}

The solution to $\sqrt{x} = 5$ is $x=25$

Now suppose we want to evaluate the sum
$\displaystyle\sum\limits_{i=0}^{n} i^3$.
We also have

$$\int_{0}^{2\pi} \sin{t}dt = 0$$
$$\int_{0}^{2\pi} \int_{0}^{2\pi}\sin{p} \sin{q}dp dq = 0$$
$$\alpha+\beta+\gamma+\delta \rightarrow 0$$

$$\mathit{x + y - z =3}$$
$$x$$


%nth root
$$\sqrt[n]{x}$$

%continued fractions
$$
\cfrac{2}{1+\cfrac{2}{1+\cfrac{2}{1+\cfrac{2}{1}}}} 
$$

$$\cos 90^\circ = 0$$


$$\dbinom{n-1}{r-1}$$

$$\overline{a+bi}$$

\begin{align*}
9 & \equiv3\bmod{6}\\	
9 & \equiv3\pmod{6}\\
9 & \equiv3\mod{6}\\
9 & \equiv3\pod{6}
\end{align*}

Now to deal with matrices

\section{Graphics Part}

We can introduce .jpg/.pdf/.png/.gif images in our document after saving it in the same directory.

\includegraphics[width=3in]{myimg2.png}


\begin{center}
\includegraphics[angle=45,scale=0.25]{myimg.jpg}
\end{center}

\section{Defining}
	\subsection{newcommand}


%within the first curly bracket name your command
%within the second curly bracket define your command
%within the square bracket tell number of arguments

\newcommand{\txt}[1]{This is writing text}

\newcommand{\vc}[1]{\underline{#1}}

\newcommand{\beg}{\begin{itemize}}
\newcommand{\cls}{\end{itemize}}



\txt\\
$$\vc{a}^2 + \vc{b}^2 = \vc{c}^2$$
\beg
\item cat
\item dog
\item rat
\cls


\newcommand{\newfrac}[2]{\dfrac{#1}{#2} + \dfrac{#2}{#1}}

$\newfrac{2}{3}$


	\subsection{renewcommand}


Has to be used before the use of the command.

\section{Tables and Matrices}

\begin{tabular}{|c|c|c|c|c|}
\hline
$x$ & 1 & 2 & 3 & 4 \\
$f(x)$ & 1 & 4 & 9 & 16 \\ 
\end{tabular}


%l signifies left justified
%r signifies left justified
%c signifies left centered
%p{} used to specify length


\begin{table}[h]

\begin{tabular}{|l|c|c|p{5cm}|c|}
\hline
Name & Lap1 & Lap2  & Result & Money \\ \hline  \hline
Sary & 12s & 13.5s & Won on a very narrow margin as we can see & 100 \\ \hline 
Gary & 11s & 15.5s & Lost & 0\\ \hline
\end{tabular}
\renewcommand{\tablename}{Table 4.1}

\caption{Race Results}
\label{tab:rce rslts}
\end{table}	

\[
\begin{pmatrix}
1 & 2 & 3 \\
0 & 1 & 2 \\
0 & 0 & 1 \\
0 & 0 & 1 \\
\end{pmatrix}
\]

\[
A=\begin{bmatrix}
    1 & 2 & 3 \\
	0 & 1 & 2 \\
	0 & 0 & 1 \\
	0 & 0 & 1 \\
	\end{bmatrix}
\]\\
\vspace{1in}
\begin{flushright}
\[
A=\begin{bmatrix}
    1 & 2 & 3 \\
	0 & 1 & 2 \\
	0 & 0 & 1 \\
	0 & 0 & 1 \\
	\end{bmatrix}
\]
\end{flushright}


\[
|A|=\begin{vmatrix}
		1 & 2 & 3 \\
		0 & 1 & 2 \\
		0 & 0 & 1 \\
		\end{vmatrix}
\]

\begin{align*}
&\overset{formula}{=} 1.1.1\\
\end{align*}   		
\begin{align*}
&=1\\
\end{align*}


$\begin{matrix}
1 & 2 & 3 \\
0 & 1 & 2 \\
0 & 0 & 1 \\
0 & 0 & 1 \\
\end{matrix}$

$
\begin{Vmatrix}
1 & 2 & 3 \\
0 & 1 & 2 \\
0 & 0 & 1 \\
0 & 0 & 1 \\
\end{Vmatrix}
$


\section{Boxes}
\subsection{Normal}
I can create basic boxes for text \makebox[1in]{like this}. Notice that there's a 1in wide space with `like this' in the middle of it.\\
If I want to put a box around the text, I can use a frame box. The result looks \framebox[1in]{like this}.\\
I can also justify the text to the right within a box \framebox[1in][r]{like so}
or \framebox[1in][l]{like so}.\\
We can also use quick versions of these. We can just \mbox{do this} or \fbox{this} to create a quick box that's exactly the size of what we put in it.

\subsection{Advanced}

\parbox[b]{2in}{I like using parbox to create funny little boxes of text all over my page.Aligned with bottom}
CURRENT LINE. \,
\parbox[t]{2in}{Aligned with top.}

\parbox{1.5in}{Centered.} CURRENT LINE \; 
\parbox{2.5in}{You probably got a few Overfull or Underfull warnings when you typeset this. Sometimes narrow boxes will do that; if you're happy with the output, don't sweat it.}

\section{color}

\colorbox{cyan}{i am getting a hang of it}


\fcolorbox{red}{green}{This is colourful :P}


\newpage
\pagecolor{cyan}


Wow its really blue

\end{document}

